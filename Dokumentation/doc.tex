\documentclass{article}
\usepackage[margin=1in]{geometry}
\author{Iúri Figueiredo Archer}
\date{\today}
\title{Documentation zum VRML Projekt\\\large{Computergrafik WS21/22}}


\begin{document}

\maketitle

\section*{Einleitung und Werkzeuge}

Dieses Dokument dient sowohl als Dokumentation für die Abgabe wie auch
als meine eigene Referenz um die Abgabe zu erleichtern. Desweiteren
nutze ich die Gelegenheit auch um meine \LaTeX\ Kenntnisse zu
erfrischen.

Es wurden folgende Programme benutzt, um das VRML Projekt zu
realizieren.
\begin{itemize}
\item \textit{Visual Studio Code} für VRML und Python
\item \textit{vim}, \LaTeX\ für die Dokumentation
\item \textit{view3dscene} und \textit{Cortona3D} für die
Visualisierung der VRML-Welt
\end{itemize}

\textit{Cortona3D} war leider notwendig, da \textit{view3dscene} noch
keine Skripte unterstützt (weder JS noch VRML-Script). 
Dies stellte sich als Herausforderung, da Cortona3D nur von Internet
Explorer unterstützt wird und sonst keinem anderen Browser - was mir
als Linux Nutzer besonders viel Spaß bereitet hat.



\section{Aufgaben}



\end{document}
